\important{Screw Strengths (December 12, 2024)}
\chapterauthor{Ian Smith}
\info{Ian Smith}{Screw Strengths}{December 12, 2024}
\section*{Screw Strengths: Fasteners Calculation Notes}

\subsection*{Formulas and Definitions}
The following equations and definitions are used for fastener calculations:

\begin{align*}
    F_i &= 0.75 \times A_t \times S_p \\
    S_p &= 0.85 \times S_y \\
    \Delta &= \frac{F_i \cdot l}{A \cdot E}
\end{align*}

Where:
\begin{itemize}
    \item $F_i$ = preload
    \item $A_t$ = tensile stress area (use tables to find values)
    \item $S_p$ = proof strength
    \item $S_y$ = yield strength
    \item $\Delta$ = change in length of the bolt
    \item $l$ = bolt length
    \item $A$ = bolt area
    \item $E$ = modulus of elasticity
\end{itemize}

Torque at the wrench can be calculated as:
\[
T = 0.2 \cdot F_i \cdot d
\]
Where:
\begin{itemize}
    \item $T$ = torque at the wrench
    \item $d$ = nominal bolt diameter
\end{itemize}

\subsection*{Thread Tensile-Stress Area}
The tensile-stress area of the thread is given by:
\[
A_s = \frac{\pi}{4} \left( \frac{(d - 1.299038 \cdot P) + (d - 0.649519 \cdot P)}{2} \right)^2
\]
Where:
\begin{itemize}
    \item $d$ = nominal diameter
    \item $P$ = number of threads per inch
\end{itemize}

\subsection*{Relating Torque to Clamping Force}
\[
T_f = K \cdot F_f \cdot d
\]
Where:
\begin{itemize}
    \item $T_f$ = torque applied
    \item $F_f$ = clamping force
    \item $K$ = torque coefficient
\end{itemize}

The torque coefficient $K$ is given by:
\[
K = \frac{1}{2d} \left( \frac{P}{\pi} + \mu_s d_2 \sec \alpha' + \mu_w D_w \right)
\]
Where:
\begin{itemize}
    \item $P$ = thread pitch
    \item $\mu_s$ = coefficient of friction (from tables)
    \item $d_2$ = pitch diameter of the thread
    \item $\mu_w$ = coefficient of friction between bearing surfaces
    \item $D_w$ = equivalent diameter of friction torque bearing surfaces
\end{itemize}

The equivalent diameter of the friction torque bearing surfaces, $D_w$, is given by:
\[
D_w = \frac{2}{3} \cdot \frac{D_o^3 - D_i^3}{D_o^2 - D_i^2}
\]
Where:
\begin{itemize}
    \item $D_o$ = outside diameter of the bearing surface contact area
    \item $D_i$ = inside diameter of the bearing surface contact area
\end{itemize}

\subsection*{Example Calculations}

\subsubsection*{Material: 1035 Carbon Steel (1" 8-32 UNC Machine Screw)}
Using normalized 1035 carbon steel with the following properties:
\begin{itemize}
    \item Tensile Strength: 75,000 psi
    \item Yield Strength: 50,000 psi
    \item Young’s Modulus: 29,000 ksi
\end{itemize}

\textbf{Step 1: Calculate Proof Strength}
\[
S_p = 0.85 \cdot S_y = 42,500 \, \text{psi}
\]

\textbf{Step 2: Calculate Tensile-Stress Area}
\[
A_s = \frac{\pi}{4} \left( \frac{(d - 1.299038 \cdot P) + (d - 0.649519 \cdot P)}{2} \right)^2
\]
\[
A_s = 0.014367 \, \text{in}^2
\]

\textbf{Step 3: Calculate Preload}
\[
F_i = 0.75 \cdot A_s \cdot S_p = 457.948 \, \text{lbs}
\]

\textbf{Step 4: Calculate Change in Length}
\[
\Delta = \frac{F_i \cdot l}{A \cdot E}
\]
\[
\Delta = 1.8688 \cdot 10^{-4} \, \text{in}
\]

\subsubsection*{Material: Aluminum Screw}
Using a conservative yield strength of 10,000 psi:
\begin{itemize}
    \item $S_p = 8,500 \, \text{psi}$
    \item $A_s = 0.0144 \, \text{in}^2$
    \item $F_i = 91.8 \, \text{lbs}$
    \item Young’s Modulus: $10,000 \, \text{ksi}$
\end{itemize}

\textbf{Change in Length:}
\[
\Delta = 0.0001086 \, \text{in}
\]

\subsubsection*{Material: Nylon Screw}
Using a yield strength of 6,000 psi:
\begin{itemize}
    \item $S_p = 5,100 \, \text{psi}$
    \item $A_s = 0.0144 \, \text{in}^2$
    \item $F_i = 54.9538 \, \text{lbs}$
    \item Young’s Modulus: $189 \, \text{ksi}$
\end{itemize}

\textbf{Change in Length:}
\[
\Delta = 3.441 \cdot 10^{-3} \, \text{in}
\]

\subsection*{References}
\begin{itemize}
    \item \href{https://www.machiningdoctor.com/threadinfo/?tid=15}{Machining Doctor Thread Info}
    \item \href{https://matweb.com/search/DataSheet.aspx?MatGUID=8d78f3cfcb6f49d595896ce6ce6a2ef1&ckck=1}{Material Data Sheet}
\end{itemize}
