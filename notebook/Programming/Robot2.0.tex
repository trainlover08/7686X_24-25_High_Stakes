\white{Coding Robot V2.0}
\chapterauthor{Ian Smith}
\info{Ian Smith}{Coding Robot V2.0}{January 13, 2025}
\section*{Coding Required Criterion}
\begin{itemize}
    \item Written in C++ using BLRS pros.
    \item Utilizes advanced algorithms to achieve quick and precise motions.
    \item Reliably provides the correct intended functionality of the robot.
\end{itemize}

\section*{Coding Optional Criterion}
\begin{itemize}
    \item Modular.
    \item Uses an easy-to-read file system.
    \item Handles complexity easily.
    \item Thoroughly commented and documented.
    \item Utilizes RTOS infrastructure to maximize CPU capabilities.
\end{itemize}

\section*{Software System Overview}
\begin{itemize}
    \item Sensors are passed in at the top level by reference and accessed for data within specific task functions and classes.
    \item Data enters the “black box” and gets passed back out as motion outputs.
    \item Motion instructions are acted upon by the robot.
\end{itemize}

\section*{System Architecture}
\begin{itemize}
    \item Uses a “building block” setup with universal classes.
    \item Basic classes are used to break down complex tasks such as color sorting and user interface.
    \item Object initialization for sensors, motors, and base classes from other libraries and namespaces.
    \item Mechanism classes are created for the intake and arm.
    \item Specific subsystem tasks are defined with adequate interlocks using the classes built.
    \item Functions are called as tasks within the proper locations in field control runtime protocol.
\end{itemize}

\section*{External Libraries}
\begin{itemize}
    \item \textbf{Robodash} — Supports \texttt{lvgl} (C graphics display for microcontrollers) for VEX VRC and provides a solid foundation for creating UI visuals on the brain.
    \item \textbf{LemLib} — An open-source motion algorithm library that includes support for PID, boomerang, odometry, pure pursuit, and additional algorithms.
\end{itemize}
