\important{Innovate Award Submission Information Form}
\textbf{Instructions for team}: Please fill out all information, printing clearly. For in-person notebooks, please place this page either inside the front cover of the team’s notebook or placed as the last entry in the notebook when submitting it for judging. In the case of digital notebooks, a picture of the form can be uploaded and placed either at the beginning of the digital notebook, after the Table of Contents, or entered as the last entry in the notebook. Teams may only submit one aspect of their design to be considered for this award at each event.

\vspace{1cm}
\textbf{Full Team Number}: 7686X 

\vspace{1cm}
\textbf{Brief Description of the novel aspect of the team’s design being submitted}:

A Reinforcement machine learning library made completely from scratch, with the help of properly sourced citations. This algorithm is created to be used during Skills. 
\vspace{1cm}


\textbf{Identify the page numbers and/or the section(s) where documentation of the
development of this aspect can be found}: 
\blueref{Artificial-Intelligence-Library}{Artificial Intelligence Library}  and \blueref{AI-Improvments (December 18, 2024)}{AI Improvements (December 18, 2024)}


\important{Planning, EDP, \& Setup (May 1, 2024)}
\label{Planning,-EDP,-&-Setup}
\chapterauthor{Caleb Bachmeier}
\info{Caleb Bachmeier}{Planning, EDP, \& Setup}{May 1,2024}
\textbf{Goal}: Breakdown how this notebook will be written


    \section*{How To Read This Notebook}
    
    \begin{itemize}
        \item Most entries will start with a goal for the entry
        \item Every entry will always start with a new chapter
        \item Every different topic in an entry will always have a subsection 
        \item Every figure will always relate to an image within in the notebook 
        \item Important information in this notebook will always be \sethlcolor{Highlighter}\hl{highlighted.}
        \item Engineering Design Process (EDP) will always be highlight to a specific color relating to the different step of the EDP (More on this in the next section, 3.2: EDP)
    \end{itemize}
\section*{EDP}
Engineering Design Process (commonly refered to as EDP throughout this notebook) is a very important part of VEX Robotics but also engineering in general. The most recent EDP that VEX follows is 


\begin{itemize}

    \item \sethlcolor{Identify}\hl{Identify the Challenge \& Set Goals}
    
    This step involves understanding the problem you're trying to solve and setting clear objectives for your project.
    
    \item \sethlcolor{Brainstorm}\hl{Brainstorm \& Diagram}
    
    Brainstorming allows you to generate ideas and explore different approaches to solving the challenge. Diagramming helps visualize concepts and organize thoughts.
    
    \item \sethlcolor{Solution}\hl{Choose a Solution \& Make a Plan}
    
    After evaluating various ideas, select the most promising solution and develop a detailed plan to execute it effectively.
    
    \item \sethlcolor{Build}\hl{Build \& Program}
    
    This phase involves constructing the physical components of your solution and programming any necessary software or algorithms.
    
    \item \sethlcolor{Test}\hl{Test the Solution}
    
    Testing is crucial to ensure that your solution functions as intended and meets the established goals. This step may involve iterative refinement.
    
    \item \sethlcolor{Repeat}\hl{Repeat the Design Process}
    
    Design is an iterative process, and it's essential to review, refine, and iterate on your solution based on feedback and new insights. We will make a chapter for repeating the design process when we revisit or change a specific part of the robot using the design process. 
    
\end{itemize}
\section*{Highlighting}
    While highlighting of EDP has already been mentioned, there are many other use of highlighting in this notebook:
    \begin{itemize}
        \item \sethlcolor{Analysis}\hl{Team / Tournament Analysis} 
        
        Team or tournament analysis will be highlighted pink in the Table of Contents following this page. Team analysis' will be documented for both teams that win each tournament we are in, for both teams at every Signature Event this year, and for the occasional robot the team finds online.
        
        Team analysis chapters include, but is not limited to:
            \begin{itemize}
                \item Team name
                \item What strategies the team employed 
                \item What type of robot the team created 
                \item What we can learn from the team
            \end{itemize}
        
        Tournament analysis chapters include, but is not limited to:
            \begin{itemize}
                \item Tournament name
                \item Who won the tournament
                \item What we learned from the tournament
                \item How we can improve from this tournament
            \end{itemize}
        \item \sethlcolor{Highlighter}\hl{Miscellaneous} 
        Miscellaneous items include, but are not limited to 
        \begin{itemize}
            \item Innovate Submission
            \item A major update in robot CAD, building, or programming
            \item Something that cannot be missed nor overlooked by judges
        \end{itemize}
        Will always be highlighted for convince for both the team and the judge grading our notebook
    \end{itemize}
\section*{Hyperlinks}
    Hyperlinks can be used to link to different parts of the same document (like sections or figures), to other local documents. When the document is viewed electronically, you can click on the hyperlink to go directly to the linked content. Hyperlinks will be used for the following items:
    \begin{itemize}
        \item Table of Contents 
        
        \sethlcolor{Highlighter}\hl{The Table of Contents after this chapter makes use of hyperlinks. This means that if you are reading this notebook online then you can click the chapter/section in the Table of Contents and be taken to that exact page within the notebook}
        \item Team Members Mentioned 
        
        Any team member mentioned will have a number next to their name relating to their section in the chapter: \blueref{team-bios}{Team Biographies} (\textless--- this is an example of a hyperlink) Any hyperlink relating to a figure or team member will have a blue underline to ensure that judges understand where a hyperlink is.
        \item Figures Mentioned

        Any figures mentioned will also have a number next to it relating to whatever the figure label is. Any hyperlink relating to a figure or team member will have a blue underline to ensure that judges understand where a hyperlink is.
    \end{itemize}
    \href{https://github.com/trainlover08/7686X_24-25_High_Stakes}{The following link is a hyperlink to the 7686X Github, in the Github you can find all of the code we have written thus far.} Although all of our code is in \blueref{Appendix A}{Appendix A.}
\pagebreak