\identify{Identify the Challenge \& Set Goals: MACHINING AND STRESS CALCULATIONS (November 18, 2024)}
\info{Ian Smith}{MACHINING AND STRESS CALCULATIONS}{November 18, 2024}
\chapterauthor{Ian Smith}
\textbf{Goal}: We will identify an objective for our robot so that we can address it and build an effective solution
\section*{Problem Statement}
Often, we don't know the minimum axle size required when building. As a result, we tend to overbuild, using High-Strength Shafts (SKU: 276-7465) "\cite{vexRobotics}" in situations where they aren't necessary, out of concern for exceeding the torsional load of a low-strength shaft.
\section*{Solution Goals}
We would love to reduce the material usage, this would result in significant weight savings.
\brainstorm{Brainstorm \& Diagram: MACHINING AND STRESS CALCULATIONS (November 18, 2024)}
\info{Ian Smith}{MACHINING AND STRESS CALCULATIONS}{November 18, 2024}
\chapterauthor{Ian Smith}
\section*{Possible Solutions}
\begin{itemize}
	\item \textbf{Guess and Check}: The quickest and simplest approach but has the potential for catastrophic failure.
	\item \textbf{Mathematical Analysis}: Involves deriving parts around calculated stresses. This approach is more complex and time-intensive.
	\item \textbf{Stress Analysis in Fusion 360}: Offers dynamic, accurate calculations and is easy to apply on a per-part basis.
\end{itemize}
\solution{Choose a Solution: MACHINING AND STRESS CALCULATIONS (November 18, 2024)}
\info{Ian Smith}{MACHINING AND STRESS CALCULATIONS}{November 18, 2024}
\chapterauthor{Ian Smith}
\section*{Choose a Solution}
\begin{itemize}
	\item Guess and Check
	\item Mathematical Solution
	\item CAD Analysis
\end{itemize}

\renewcommand{\arraystretch}{1.85} % Change this value as needed
\begin{table}[htb!]
\centering
\begin{tabular}{|>{\centering\arraybackslash}m{1.85cm}|>{\centering\arraybackslash}m{1.85cm}|>{\centering\arraybackslash}m{1.85cm}|>{\centering\arraybackslash}m{1.85cm}|>{\centering\arraybackslash}m{1.85cm}|>{\centering\arraybackslash}m{1.85cm}|>{\centering\arraybackslash}m{1.85cm}|}
\hline
\textbf{Scale 1 - 10} & \textbf{Weight} & \textbf{Accuracy} & \textbf{Universality} & \textbf{Total} \tabularnewline
\hline
Weight & x2 & x1 & x3 & \tabularnewline
\hline
Guess and Check & 3 & 10 & 6 & 34\tabularnewline
\hline
Mathematical Solution & 8 & 6 & 10 & 54\tabularnewline
\hline
CAD Analysis & 10 & 7 & 7 & 48\tabularnewline
\hline
\end{tabular}
\caption{Machining Decision Matrix}
\label{tab:drive-matrix}
\end{table}
\renewcommand{\arraystretch}{1.85} % Reset to default

I initially expected CAD analysis to be the best option. However, after weighing the criteria, the mathematical solution emerged as the most favorable approach. 

\section*{Make a Plan}
For this, I will rely exclusively on \machinists


\build{Build: MACHINING AND STRESS CALCULATIONS (November 18, 2024)}
\info{Ian Smith}{MACHINING AND STRESS CALCULATIONS}{November 18, 2024}
\chapterauthor{Ian Smith}
\section*{Building}
\machinists provides the following formula:

\[
T = S_s \cdot Z_p
\]

Where:
\begin{itemize}
	\item \(T\): Torsional moment (PSI)
	\item \(S_s\): Allowable shearing stress (PSI)
	\item \(Z_p\): Polar section modulus
\end{itemize}

To calculate \(Z_p\), the formula for a square cross-section is:

\[
Z_p = 0.208a^3
\]

Where:
\begin{itemize}
	\item \(a\): Width of the square shaft
\end{itemize}

Next, we determine whether \(T\) is sufficient for the RPM and horsepower using:

\[
T = \frac{63000P}{N}
\]

Where:
\begin{itemize}
	\item \(N\): Shaft RPM
	\item \(P\): Power applied to the shaft (HP)
\end{itemize}

Finally, the maximum allowable shearing stress \(S_s\) can be calculated as follows:

\[
S_w = \frac{S_m}{f_s}
\]

Where:
\begin{itemize}
	\item \(S_m\): Material strength
	\item \(f_s\): Factor of safety
\end{itemize}

The working stress is adjusted by the stress concentration factor:

\[
\sigma = \frac{S_w}{K_t}
\]

Where:
\begin{itemize}
	\item \(K_t\): Stress concentration factor (based on material and load type)
\end{itemize}

Torsional stress (\(\tau\)) can be calculated as:

\[
\tau = \frac{T}{Z_p}
\]

The total stress on a member is given by:

\[
Z = \frac{L_a \cdot \tau}{\sigma}
\]

Where:
\begin{itemize}
	\item \(L_a\): Axle length
\end{itemize}
\test{Test the Solution: MACHINING AND STRESS CALCULATIONS (November 18, 2024)}
\info{Ian Smith}{MACHINING AND STRESS CALCULATIONS}{November 18, 2024}
\chapterauthor{Ian Smith}
\section*{Test the Solution}
For this test, consider a 100 RPM drive axle powered by a single motor at 2.12 Nm torque using a $\frac{1}{8}$
inch shaft.

Let:
\begin{itemize}
	\item \(K_t = 2.0\)
	\item \(f_s = 2.0\)
	\item \(L_a = 12 \, \text{inches}\)
	\item \(a = \frac{1}{8} \, \text{inch}\)
	\item \(S_m = 60000 \, \text{PSI}\)
	\item \(N = 100\)
	\item \(\tau = 2.12 \, \text{Nm}\)
\end{itemize}

First, convert torque to inch-pounds:

\[
\tau = 2.12 \, \text{Nm} = 0.7376 \cdot 2.12 = 1.5637 \, \text{inch-lbs}
\]

Next, calculate power (\(P\)) in horsepower:

\[
P = \frac{\tau \cdot N}{5252} = \frac{1.5637 \cdot 100}{5252} = 0.029774 \, \text{HP}
\]

Calculate \(S_w\):

\[
S_w = \frac{S_m}{f_s} = \frac{60000}{2} = 30000
\]

Calculate \(\sigma\):

\[
\sigma = \frac{S_w}{K_t} = \frac{30000}{2.0} = 15000
\]

Calculate \(T\):

\[
T = \frac{63000P}{N} = \frac{63000(0.029774)}{100} = 18.7476
\]

Calculate \(Z_p\) from \(a\):

\[
Z_p = 0.208a^3 = 0.208 \left(\frac{1}{8}\right)^3 = 0.00040625
\]

Calculate \(\tau\):

\[
\tau = \frac{T}{Z_p} = \frac{18.7476}{0.00040625} = 4614.7938 \, \text{inch-lbs}
\]

Finally, solve for \(Z\):

\[
Z = \frac{L_a \cdot \tau}{\sigma} = \frac{12 \cdot 4614.7938}{15000} = 3.6918 \, \text{inch}^3
\]

Determine the minimum bar size:

\[
a = \sqrt[3]{Z \cdot 6} = \sqrt[3]{3.6918 \cdot 6} = \sqrt[3]{22.1508} = 2.8084 \, \text{inches}
\]

This result appears incorrect due to ambiguities in defining \(a\). Further adjustments to calculations are necessary, as detailed in subsequent sections.

\subsection*{Final Derived Equations}
\[
\tau = \text{nm} \cdot 0.7376
\]

\[
P = \frac{\tau \cdot N}{5252}
\]

\[
S_w = \frac{S_m}{f_s}
\]

\[
\sigma = \frac{S_w}{K_t}
\]

\[
T = \frac{63000P}{N}
\]

\[
Z_p = 0.208a_{\text{initial}}^3 \cdot S_s
\]

\[
\tau = \frac{T}{Z_p}
\]

\[
Z = \frac{L_a \cdot \tau}{\sigma}
\]

\[
a_{\text{final}} = \sqrt[3]{Z \cdot 6}
\]

The test has succeeded. This framework allows us to accurately determine whether an axle will fail or succeed based on input parameters.
